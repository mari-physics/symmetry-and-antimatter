%----------------------------------------
% Basic project setup
%----------------------------------------
\documentclass[a4paper,12pt]{article} % Defines document type and size

%----------------------------------------
% Useful packages
%----------------------------------------
\usepackage{amsmath}  % For advanced mathematical equations
\usepackage{graphicx} % To include figures and images
\usepackage{physics}  % For physics-related symbols and notations

%----------------------------------------
% Title and author info
%----------------------------------------
\title{Mathematical Symmetry in Physics}  % Project title
\author{Mari Harbi}                   % Your name
\date{\today}                             % Current date

%----------------------------------------
% Beginning of the document
%----------------------------------------
\begin{document}

\maketitle % Generates the title section

%----------------------------------------
% Section: Introduction
%----------------------------------------
\section*{Introduction}
Symmetry plays a fundamental role in modern physics. 
It helps describe conservation laws and the structure of physical theories. 
Understanding symmetry provides insight into why the universe behaves the way it does.

%----------------------------------------
% Section: Types of Symmetry
%----------------------------------------
\section*{Types of Symmetry}
In physics, symmetry can appear in several forms:
\begin{itemize}
    \item \textbf{Translational symmetry:} The laws of physics remain the same regardless of position in space.
    \item \textbf{Rotational symmetry:} Physical systems behave the same when rotated.
    \item \textbf{Mirror (Parity) symmetry:} Refers to the invariance of a system under spatial reflection.
\end{itemize}

%----------------------------------------
% Section: A Simple Mathematical Example
%----------------------------------------
\section*{A Simple Mathematical Example}
Consider a function that remains unchanged when $x$ is replaced by $-x$:
\[
f(x) = f(-x)
\]
Such a function is said to be \textit{even}, showing mirror symmetry about the $y$-axis.

%----------------------------------------
% Section: Conclusion
%----------------------------------------
\section*{Conclusion}
Symmetry is not merely a mathematical curiosity—it is a guiding principle in theoretical physics 
that underpins the laws governing matter and energy.

%----------------------------------------
% End of the document
%----------------------------------------
\end{document}
